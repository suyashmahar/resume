\documentclass{article}

\usepackage[includeheadfoot, left=0.5in, right=0.675in,  top=0.3in,  bottom=0.3in]{geometry}

% For \text command in mathmode
\usepackage{amsmath}

% Change the line spacing
\usepackage{setspace} 

% For changing item separation in itemize environment
\usepackage{enumitem}

% For named colors
\usepackage{xcolor}

% For bigger cdots
\usepackage{graphicx}

% For custom font
\usepackage{fontawesome}

% Disable page numbering
\usepackage{lastpage}

% For fancy headers
\usepackage{fancyhdr}

% Change the default font
\usepackage{fontspec}
\setmainfont{Fontin}[ 
Extension = .otf,
UprightFont = *-Regular,
BoldFont = *-Bold,
ItalicFont = *-Italic,
]

\usepackage{sansmath}

% Define color for the links
\definecolor{cobalt}{rgb}{0.0, 0.28, 0.67} 

% Setup the hyperlinks
\usepackage{hyperref}
\hypersetup{%
  colorlinks=true,% hyperlinks will be black
  urlcolor=cobalt,% hyperlink borders will be red
}
\usepackage{soul}
% For bigger cdots
\makeatletter
\newcommand*{\bigcdot}{}% Check if undefined
\DeclareRobustCommand*{\bigcdot}{%
  \mathbin{\mathpalette\bigcdot@{}}%
}
\newcommand*{\bigcdot@scalefactor}{.5}
\newcommand*{\bigcdot@widthfactor}{1.15}
\newcommand*{\bigcdot@}[2]{%
  % #1: math style
  % #2: unused
  \sbox0{$#1\vcenter{}$}% math axis
  \sbox2{$#1\cdot\m@th$}%
  \hbox to \bigcdot@widthfactor\wd2{%
    \hfil
    \raise\ht0\hbox{%
      \scalebox{\bigcdot@scalefactor}{%
        \lower\ht0\hbox{$#1\bullet\m@th$}%
      }%
    }%
    \hfil
  }%
}
\makeatother

% Item separation for work experience enumeration
\newlength{\workexitemsep}
\setlength{\workexitemsep}{-1.6mm} 

% Item separation for project description enumeration
\newlength{\projitemsep}
\setlength{\projitemsep}{-1.2mm} 

% Separation between section and first line
\newlength{\sectionbottommargin}
\setlength{\sectionbottommargin}{1mm} 

% First column of experience
\newlength{\durlen}
\setlength{\durlen}{0.08\textwidth}

% Second column of experience
\newlength{\deslen}
\setlength{\deslen}{0.86\textwidth}

% Empty environment
\newenvironment{nothing}{}

%% Define macros
% For declaring section title with a hrule
\newcommand{\sectiontitle}[1]{
  \textbf{\Large{\sffamily #1}}
  \begin{flushright}
      \vspace{-0.48cm} 
      \rule{0.975\textwidth}{1.1pt}
  \end{flushright}
  \vspace{-0.1cm} 
}

% For declaring subsection title
\newcommand{\subsectiontitle}[1]{
  \textbf{\large{\sffamily\hspace{0cm} #1}}
  \begin{center}
      \vspace{-0.55cm} 
      % \rule{0.945\textwidth}{1.1pt} 
  \end{center}
  \vspace{-0.1cm} 
}

% Define color for the publication text
\definecolor{pubcolor}{gray}{0.25} 

% For primary author in publication list
\newcommand{\pubprimauthor}[1]{\color{black}{\textbf{#1}}\color{pubcolor}}

% For primary author in publication list
\newcommand{\pubtitle}[1]{\textbf{\color{black}{#1}}\color{pubcolor}}

\usepackage{hyperref}
\usepackage{etoolbox} 

% Define the header and footer
\pagestyle{fancy}
\fancyhf{}
\rhead{
  \sffamily\it \textcolor{gray}{Resume - Suyash Mahar}
}
\rfoot{\sffamily\it \textcolor{gray}{Page {\thepage} of \begin{NoHyper}\pageref{LastPage}\end{NoHyper}}}
  \renewcommand{\headrulewidth}{0pt}

% Different style for the first page
\fancypagestyle{firstpage}{%
  \fancyhf{}% clear default for head and foot
  \rfoot{\sffamily\it \textcolor{gray}{Page {\sffamily\thepage} of \begin{NoHyper}\pageref{LastPage}\end{NoHyper}}}
  \renewcommand{\headrulewidth}{0pt}
}

\vspace{-2.5cm}

\begin{document}

\thispagestyle{firstpage}

% Header
\begin{singlespace}
    % Name
    \begin{center}
        \textbf{\Huge \sffamily Suyash Mahar}
    \end{center}

    % Address
    \begin{center}
        \vspace{-0.1cm}
        10702 Dabney Dr Apt 95, San Diego, CA - 92126
    \end{center}

    % Communication
    \begin{center}
        \vspace{-0.2cm}
        \faPhone\ +1 412 294 8345 $\bigcdot$
        \faEnvelope\ \href{mailto:smahar@ucsd.edu}
        {smahar@ucsd.edu} $\bigcdot$ \faGlobe\ \href{https://suyashmahar.com}{suyashmahar.com} %$\bigcdot$ \faGithub\ \href{https://github.com/suyashmahar}{github/suyashmahar}
    \end{center}
\end{singlespace}

%% == %% == %% == %% == %% == %% == %% == %%
\sectiontitle{Education}

\begin{tabular}{p{0.24\textwidth} p{0.6\textwidth}}
  \hfill \textit{Doctor of Philosophy}           & \large\textbf{University of California San Diego} \\
  \hfill \footnotesize 2020-Present              & Computer Science and Engineering (\textit{Advisor: Prof. Steven Swanson)}\\[2mm]
  \hfill \textit{Bachelor of Technology}         & \large \textbf{Indian Institute of Technology (IIT) Roorkee} \\
  \hfill \footnotesize 2016-2020                 & Electronics and Communication Engineering\\
\end{tabular} 

\vspace{2mm}

%% == %% == %% == %% == %% == %% == %% == %%
\sectiontitle{Research Interests}

\begin{tabular}{p{0.908\textwidth} p{0.01\textwidth}} 
  System and software support for emerging memory/storage hierarchy in datacenters.&\\
\end{tabular}
\vspace{4mm}

%% == %% == %% == %% == %% == %% == %% == %%
\sectiontitle{Industry Experience}

\begin{tabular}{p{\durlen}p{\deslen}}
  \centering June 2023       & \textbf{Google's Cloud Workload Acceleration} \hfill \textit{Google LLC} \\
  \centering -               & \textit{\href{https://techsysinfra.google/research}{Systems Research Group (SRG)}}\\ 
  \centering Dec 2023        & \vspace{-0.6cm}\begin{itemize}[itemsep=\workexitemsep]
          \item Studied Google's workload traces for hardware acceleration opportunities.
          \item Built tools for finding chain-of-accelerator patterns across the fleet.
          \item Used fleet-wide stack samples to recreate possible execution path for workloads.
          \item Found multiple examples of potential accelerator chaining across Google's codebase.
  \end{itemize}
\end{tabular}

\begin{tabular}{p{\durlen} p{\deslen}}
  \centering June 2022       & \textbf{Meta's Workload Profiling for Next Generation Processors} \hfill \textit{Meta Platforms Inc.} \\
  \centering -               & \textit{Capacity Engineering and Analysis (CEA)} \\ 
  \centering Sept 2022       & \vspace{-0.6cm}\begin{itemize}[itemsep=\workexitemsep] 
          \item Developed tools to profile for Meta's workloads to understand HW needs for the next gen servers.
          \item Performed detailed analysis of production workloads' memory and cache behavior.
          \item Showed 9.1\% IPC improvement with production traffic for new microarchitectural changes.
          \item Developed low-overhead memory access tracing tools for tracing production workloads.
          \item Pre-print of the work available on \href{https://arxiv.org/pdf/2303.08396}{arXiv}.
  \end{itemize}
\end{tabular}

\begin{tabular}{p{\durlen} p{\deslen}}
  \centering Jun 2021       & \textbf{Page Migration for Tiered Memory Systems} \hfill \textit{Intel Labs, USA} \\
  \centering -              & \textit{Systems \& Software Research (SSR)} \\ 
  \centering Sept 2021      & \vspace{-0.6cm}\begin{itemize}[itemsep=\workexitemsep] 
          \item Worked on a cache-coherent FPGA and kernel driver to migrate pages between memories.
          \item Implemented support for tracking pages and sending stats over PCIe using memory mapped regs.
          \item Evaluated the implementation to migrate pages between host memory and FPGA attached DRAM.
  \end{itemize}
\end{tabular}

\vspace{-0.2cm}


\sectiontitle{Patents} 
\vspace{-3mm}

\begin{nothing}
    \vspace{\sectionbottommargin} 
    
    \ifdefined\resume
        \vspace{-0.2cm} 
    \fi 

    \begin{enumerate}
            \item H. Wang, \textbf{Suyash Mahar}, A. Dhanotia, W. Su\\
        \textbf{Sharing Translation Lookaside Buffer Entries Across Central Processing Unit 
          Cores and Process} \\
        {\it\footnotesize \textbf{Patent Pending}, Application No. 18080574, U.S. Patent and Trademark Office}
        
    \end{enumerate}
\end{nothing}


\sectiontitle{Publications} 

\vspace{-3mm}
% \fi
\begin{nothing}
    \vspace{\sectionbottommargin}
    \color{pubcolor}
    
    \ifdefined\resume
        \vspace{-0.2cm} 
    \fi 

    \begin{enumerate}
            \item \pubprimauthor{Suyash Mahar}, E. Hajyasini, S. Lee, Z. Zhang, M. Shen, S. Swanson\\
        \pubtitle{Telepathic Datacenters: Fast RPCs using Shared CXL Memory} \\
        {\footnotesize \textit{In Progress}, Preprint: \href{https://arxiv.org/abs/2408.11325}{\textit{arXiv preprint arXiv:2408.11325, 2024}}}
            \item Yi Xu, \pubprimauthor{Suyash Mahar}, Z. Liu, M. Shen, S. Swanson\\
        \pubtitle{CXL Shared Memory Programming: Barely Distributed and Almost Persistent} \\
        {\footnotesize \textit{In Progress}, Preprint: \href{https://arxiv.org/abs/2405.19626}{\textit{arXiv preprint arXiv:2405.19626, 2024}}}
            \item \pubprimauthor{Suyash Mahar}, H. Wang, W. Shu, A. Dhanotia\\
        \pubtitle{Workload Behavior Driven Memory Subsystem Design for Hyperscale} \\
        {\footnotesize \href{https://arxiv.org/abs/2303.08396}{\textit{arXiv preprint arXiv:2303.08396, 2023}}}
            \item \pubprimauthor{Suyash Mahar}, M. Shen, TJ Smith, J. Izraelevitz, S. Swanson\\
        \pubtitle{Puddles: Application-Independent Recovery and Location-Independent Data for Persistent Memory} \\
        {\footnotesize \textit{19$^\text{th}$ European Conference on Computer Systems (\textbf{EuroSys'24}) \hfill Athens, Greece}} [\href{https://dl.acm.org/doi/abs/10.1145/3627703.3629555}{ACM}]
            \item \pubprimauthor{Suyash Mahar}, M. Shen, T. Kelly, S. Swanson\\
        \pubtitle{Snapshot: Fast, Userspace Crash Consistency for CXL and PM Using msync} \\
        {\footnotesize \textit{41$^\text{st}$ IEEE International Conference on Computer Design (\textbf{ICCD'23}) \hfill Washington DC, USA}} [\href{https://www.computer.org/csdl/proceedings-article/iccd/2023/429100a495/1T97nj4RzFe}{IEEE}]
            \item \pubprimauthor{Suyash Mahar}, S. Liu, K. Seemakhupt, V. Young, S. Khan.\\
        \pubtitle{Write Prediction for Persistent Memory Systems} \\
        {\footnotesize \textit{30$^{\text{th}}$ Int'l Conf. on Parallel Architectures and Compilation Techniques (\textbf{PACT'21}) \hfill Virtual, South Korea}} [\href{https://ieeexplore.ieee.org/document/9563038/}{IEEE}]
            \item S. Liu*, \pubprimauthor{Suyash Mahar*}, B. Ray., S. Khan.\\
        \pubtitle{PMFuzz: Test Case Generation for Persistent Memory Programs} \\
        {\footnotesize \textit{26$^{\text{th}}$ Int'l Conf. on Architectural Support for Progr. Languages and Operating Systems (\textbf{ASPLOS'21}) \hfill Virtual, USA}} [\href{https://dl.acm.org/doi/abs/10.1145/3445814.3446691}{ACM}]\\
        {\footnotesize \textit{(\ * = Equal contribution authors\ )}}
            \item L. Yavits, L. Orosa, \pubprimauthor{Suyash Mahar}, J. Ferreira, O. Mutlu., R. Ginosar, M. Erez\\
        \pubtitle{Enhancing Wear-Leveling and Fault Tolerance in Resistive Memories Using Programmable Address Decoders}\\
        {\footnotesize \textit{38$^{\text{th}}$ Int'l Conf. on Computer Design (\textbf{ICCD'20})} \hfill Hartford CT, USA} [\href{https://ieeexplore.ieee.org/abstract/document/9283556}{IEEE}]
            \item D. Saxena, \pubprimauthor{Suyash Mahar}, V. Raychoudhury, J. Cao\\
        \pubtitle{Scalable, High-speed On-chip-based NDN Name Forwarding using FPGA} \\
        {\footnotesize \textit{20$^{\text{th}}$ Int'l Conf. on Distributed Computing and Networking (\textbf{ICDCN'19})} \hfill Bangalore, India} [\href{https://doi.org/10.1145/3288599.3288613}{ACM}]
    \end{enumerate}
\end{nothing}

\vspace{2mm}
% \subsectiontitle{(\textit{under review})} 
% \ifdefined\cv
%     \vspace{-5mm}
% \fi
% \begin{nothing}
%     \vspace{\sectionbottommargin} 

%     \ifdefined\resume
%         \vspace{-0.2cm} 
%     \fi 

%     \begin{enumerate}
%             \item S. Liu*, \textbf{S. Mahar*}, K. Seemakhupt, V. Young, S. Khan (* = co-first author)\\
%         \textbf{Write Prediction for Persistent Memory Systems} \\
%         {\footnotesize Submitted to the \textit{47th  International Symposium on Computer Architecture (ISCA 2020)}} %, Bangalore, India, January 04 - 07, 2019}
%             \item L. Yavits, \textbf{S. Mahar}, L. Orosa, J. D. Sanches, R. Ginosar, O. Mutlu and M. Erez \\
%         \textbf{WoLFRaM: Enhancing Wear-Leveling and Fault Tolerance in Resistive Memories Using Programmable Address Decoder}\\
%         {\footnotesize Submitted to the \textit{50th IEEE/IFIP Int. Conference on Dependable Systems and Networks (DSN 2020)} }%Vol. XX, No. x1, xxx. – xxx. xxxx.}
%     \end{enumerate}
% \end{nothing}


\ifdefined\resume
    \vspace{-0.2cm} 
\fi 

\sectiontitle{Research Projects}

\begin{enumerate}
        \item \textbf{Characterizing CXL devices and failures}\hfill{}Feb 2024 - Present\\
    \textit{UC San Diego, UIUC, and UC Berkeley}\vspace{-0.15cm}
    \begin{itemize}[itemsep=\projitemsep]
            \item Characterizing CXL memory expanders and FPGAs for a variety of workloads and microbenchmarks.
            \item Working on classifying CXL shared memory failure modes.
            \item Developing failure identification and recovery mechanisms.
    \end{itemize}
        \item \textbf{RPCool: CXL-based shared memory RPCs}\hfill{}May 2023 - Present\\
    \textit{UC San Diego}\vspace{-0.15cm}
    \begin{itemize}[itemsep=\projitemsep]
            \item Developed an RPC framework using CXL's shared memory (SHM) capabilities for datacenter applications.
            \item Implemented safety through new memory protection features while avoiding serialization and compression.
            \item Addressed scalability limitations of CXL by integrating RDMA-based communication. 
            \item Reduced round-trip RPC latency by 7.2$\times$ compared to state-of-the-art CXL RPC framework.
      \end{itemize}
    \item \textbf{System-level storage abstraction for relocatable and recoverable persistent memory}\hfill{}Dec 2020 - Dec 2021\\
      \textit{UC San Diego}\vspace{-0.15cm}
    \begin{itemize}[itemsep=\projitemsep]
            \item Implemented a new abstraction for accessing persistent memory using Puddles.
            \item Puddles use native, 8-byte pointers for pointing to persistent data while being relocatable.
            \item Puddles provide recovery as a system support, data is guaranteed to be recovered by the system after a crash.
            \item Achieved speedup of up to 1.34$\times$ compared to Intel's PMDK while enable relocation and guaranteed recovery.
    \end{itemize}

        \item \textbf{Crash-Consistency bug detection in Persistent Memory software}\hfill{}May 2020 - Aug 2020\\
    \textit{University of Virginia}\vspace{-0.15cm}
    \begin{itemize}[itemsep=\projitemsep]
            \item Implemented an end-to-end framework for testing and reporting crash-consistency bugs (CCB).
            \item Designed a fuzzer optimized for focussing on CCB prone execution paths.
            \item Found 12 new, previously unreported bugs in Intel's PMDK library.
    \end{itemize}

        \item \textbf{Optimization of hardware support for Persistent Memories} \hfill{}Aug 2019 - Dec 2019\\
    \textit{University of Virginia}\vspace{-0.15cm}
    \begin{itemize}[itemsep=\projitemsep]
             \item PM systems require various H/W support, such as encryption, compression, and wear-leveling.
             \item Worked on a software transparent way to speed them up by precomputing their results.
             \item Achieved significant performance and energy improvement across PMDK workloads.
    \end{itemize}

%     \centering Aug 2019       &  \\
%     \centering -              & \href{https://www.cs.virginia.edu/~smk9u/}{Prof. Samira Khan}, ShiftLab\\ 
%     \centering Dec 2019        & \vspace{-0.6cm}\begin{itemize}[itemsep=\workexitemsep] 
%     \end{itemize}
% \end{tabular}


% \begin{tabular}{p{\durlen} p{\deslen}}
%     \centering June 2019       & \textbf{Wear leveling in Persistent Memory} \hfill \textit{Technion, Israel} \\
%     \centering -              & Dr. Leonid Yavits and \href{http://webee.technion.ac.il/~ran/}{Prof. Ran Ginosar}, VLSI group\\ 
%     \centering July 2019      & \vspace{-0.6cm}\begin{itemize}[itemsep=\workexitemsep] 
%             \item Worked on wear-leveling in persistent memories (e.g. PCM) using programmable address decoder.
%             \item Designed architectural modification to PCM array for adding the wear leveling mechanism.
%     %             \item Developed communication interface between memory and controller for addr/data remapping.%Defined relationship between memory module and memory controller for addr/data remapping. \hl{Comment: I do not understand this point. What is ``define relationship''?}
%             \item Evaluated performance, energy and efficiency using custom written C++ and Python tools.
%             \item Lifetime improvement of 1.68x with worst case performance overhead of 3.8\% across SPEC 2006.
%     \end{itemize}
% \end{tabular}  

\end{enumerate}

\ifdefined\resume
    \vspace{-0.2cm} 
\fi 

\newpage
\sectiontitle{Outreach Activities}

\begin{tabular}{p{\durlen} p{\deslen}}
  \centering 2022 -       & \textbf{Co-founder, Students@Systems group} [\href{https://students-at-systems.org}{Website}]  \hfill \textit{} \\
  \centering Present      & \vspace{-0.6cm}\begin{itemize}[itemsep=\workexitemsep]
          \item Student led group across multiple universities, organizing panels and talks.
          \item Events focused on helping students with different aspects of PhD life and career.
      
  \end{itemize}
\end{tabular}

\vspace{-0.2cm}

\begin{tabular}{p{\durlen} p{\deslen}}
  \centering 2022 -       & \textbf{Master's Research Mentoring - UCSD}  \hfill \textit{} \\
  \centering 2024         & \vspace{-0.6cm}\begin{itemize}[itemsep=\workexitemsep]
          \item Mentored master's student Zifeng Zhang on systems research.
          \item Worked on implementing RDMA-based distributed shared memory.
      
  \end{itemize}
\end{tabular}

\vspace{-0.2cm}

\begin{tabular}{p{\durlen} p{\deslen}}
  \centering 2022 -       & \textbf{Undergraduate Research Mentoring - UCSD}  \hfill \textit{} \\
  \centering 2023         & \vspace{-0.6cm}\begin{itemize}[itemsep=\workexitemsep]
          \item Mentored undergrads, Drawin Chen and Nirmal Agnihotri on getting started with CS research.
          \item Worked with WebAssembly and persistent memory for serverless computing environment.
      
  \end{itemize}
\end{tabular}

\vspace{-0.2cm}

\begin{tabular}{p{\durlen} p{\deslen}}
  \centering 2021 -       & \textbf{Grad Women in Computing (GradWiC) Mentoring}  \hfill \textit{} \\
  \centering 2023      & \vspace{-0.6cm}\begin{itemize}[itemsep=\workexitemsep]
          \item Mentoring and monthly meetings with mentees on managing different aspects of PhD life.
      
  \end{itemize}
\end{tabular}


\sectiontitle{Volunteer Activities}

\begin{tabular}{p{\durlen} p{\deslen}}
  \centering 2022    & \hspace{0.5cm}CASA Summer DEI Reading Group\\
  \centering 2022    & \hspace{0.5cm}Artifact Evaluation Committee, OSDI and ATC 2022\\
  \centering 2022    & \hspace{0.5cm}Co-organizer, Summer Computer Architecture DEI reading series\\
  \centering 2021-22 & \hspace{0.5cm}Web chair, Computer architecture long term mentoring (CALM) [\href{http://www.comparchmentoring.org}{Website}]\\
  \centering 2021-22 & \hspace{0.5cm}UCSD CSE Visit Day Coordinator\\
\end{tabular}
    
    
\vspace{0.4cm}
    % \newpage
    % \sectiontitle{Teaching Experiences}

    % \begin{tabular}{p{\durlen} p{\deslen}}
    %     \centering Spring 18, 19 & \textbf{Digital Logic Design: Undergraduate teaching assistant}  [\href{http://suyashmahar.com/ecn104-assignments/}{repository}]  \hfill Dept. of ECE, IITR  \\
    %     \centering                & Enrolment: 85 (2018), 104 (2019)\\
        
    %     \centering                & \vspace{-0.6cm}\begin{itemize}[itemsep=\workexitemsep]
    %     \item Wrote assignments on Verilog HDL and taught basics of HDL and digital design.
        
    % \end{itemize}
    % \end{tabular}   
    
    % \ifdefined\cv
    %     \vspace{-0.2cm} 
    % \fi 
    % \ifdefined\resume
    %     \vspace{-0.4cm} 
    % \fi 

    % \begin{tabular}{p{\durlen} p{\deslen}}
    %     \centering Autumn'18       & \textbf{Object-oriented programming: Undergraduate teaching assistant} \hfill Dept. of CSE, IITR \\
    %     \centering                & Enrolment: 87\\
        
    %     \centering                & \vspace{-0.6cm}\begin{itemize}[itemsep=\workexitemsep]
    %     \item Helped freshers with little or no experience in programming to get started with basics of Java.
        
    % \end{itemize}
    % \end{tabular}
    
    % \ifdefined\cv\newpage\fi
    % \ifdefined\cv {
    % \sectiontitle{Awards and Achievements}
    
    % \begin{tabular}{p{\durlen} p{\deslen}}
    %     \centering April 2017       & Dedicated member award, National Service Scheme (IIT Roorkee) \hfill
    % \end{tabular}

    % \sectiontitle{Other Projects}

    % \ifdefined\resume
    %     \vspace{-0.2cm}
    % \fi 

    % \ifdefined\cv
    % \begin{tabular}{p{\durlen} p{\deslen}}
    %     \centering Dec. 2018       & \textbf{Real time editor for file streams in Linux} (C98) \href{https://github.com/suyashmahar/fsed}{\faGithub} \\
    %                               & \vspace{-0.6cm}\begin{itemize}[itemsep=-2mm]\item Replaces content read from a file by a process without actually changing it on the disk.
    %                                 \item Used syscall \texttt{ptrace(2)} and text manipulation to achieve real time modification of file streams.
    % \end{itemize}
    % \end{tabular}

    % \vspace{-0.2cm}

    % \fi 

    % \ifdefined\cv
    % \begin{tabular}{p{\durlen} p{\deslen}}
    %     \centering May 2018        & \textbf{URISC: Single Instruction Processor and Toolchain} (SystemVerilog, Python, Haskell) \href{https://github.com/suyashmahar/urisc}{\faGithub} \\
    %                               & \vspace{-0.6cm}\begin{itemize}[itemsep=-2mm]\item Designed and implemented a \texttt{subleq} instruction based processor.
    %                                 \item Wrote a Python and Haskell based assembler toolchain.
    %                                 \item Wrote assembler support for limited MIPS ISA, recursive macros, \texttt{include} statements and labels.
    % \end{itemize}
    % \end{tabular}

    % \vspace{-0.2cm}

    % \fi 
    
    % \begin{tabular}{p{\durlen} p{\deslen}}
    %     \centering Dec. 2017       & \textbf{.gitignore file manager for Linux} (Shell programming) \href{https://github.com/suyashmahar/pine}{\faGithub} \\
    %                               & \vspace{-0.6cm}\begin{itemize}[itemsep=-2mm]\item Project work includes writing BASH shell script. \end{itemize}
    % \end{tabular}
        
    % \ifdefined\cv
    %     \vspace{-0.2cm}
    % \fi
    % \ifdefined\resume
    %     \vspace{0.1cm}
    % \fi
 
    % \begin{tabular}{p{\durlen} p{\deslen}}
    %     \centering     2017   & \textbf{Triangle art generating views} (Java programming, Team size: 3) \href{https://github.com/sdsmdg/trianglify}{\faGithub} \\
        
    \sectiontitle{\sffamily Skills}
    
    \begin{tabular}{p{\durlen} p{\deslen}}
    \centering \textbf{Technologies} & \hspace{0.5cm}CXL, WebAssembly, RDMA, Software Transactional Memories, Intel MPK, and Gem5\\[1mm]
    \centering \textbf{Skills} & \hspace{0.5cm}Perf, Intel VTune, WebAssembly, MongoDB and Memcached Internals\\[1mm]
    \centering \textbf{Languages} & \hspace{0.5cm}C++, Python, C, Bash Script, Verilog, and \LaTeX\\[1mm]
    \end{tabular}
    
    
     
\end{document}
